% Don't touch this %%%%%%%%%%%%%%%%%%%%%%%%%%%%%%%%%%%%%%%%%%%
\documentclass[11pt]{article}
\usepackage[utf8]{inputenc}
\usepackage{fullpage}
\usepackage[left=1in,top=1in,right=1in,bottom=1in,headheight=3ex,headsep=3ex]{geometry}
\usepackage{graphicx}
\usepackage{float}
\usepackage[spanish]{babel}

\newcommand{\blankline}{\quad\pagebreak[2]}
%%%%%%%%%%%%%%%%%%%%%%%%%%%%%%%%%%%%%%%%%%%%%%%%%%%%%%%%%%%%%%

% Modify Course title, instructor name, semester here %%%%%%%%

\title{2535535: Fundamentos de Electrónica Digital}
\author{\begin{tabular} {c|c|c|c|c|c} Teórico & \textbf{4} & Teórico-Práctico & \textbf{0} & Práctico & \textbf{3}\end{tabular}
\\
\begin{tabular} {c|c|c|c|c|c} Validable & \textbf{No} & Habilitable & \textbf{No} & Clasificable & \textbf{No}\end{tabular}}
\date{2016-1}

%%%%%%%%%%%%%%%%%%%%%%%%%%%%%%%%%%%%%%%%%%%%%%%%%%%%%%%%%%%%%%

% Don't touch this %%%%%%%%%%%%%%%%%%%%%%%%%%%%%%%%%%%%%%%%%%%
\usepackage[sc]{mathpazo}
\linespread{1.05} % Palatino needs more leading (space between lines)
\usepackage[T1]{fontenc}
\usepackage[mmddyyyy]{datetime}% http://ctan.org/pkg/datetime
\usepackage{advdate}% http://ctan.org/pkg/advdate
\newdateformat{syldate}{\twodigit{\THEMONTH}/\twodigit{\THEDAY}}
\newsavebox{\MONDAY}\savebox{\MONDAY}{Mon}% Mon
\newcommand{\week}[1]{%
%  \cleardate{mydate}% Clear date
% \newdate{mydate}{\the\day}{\the\month}{\the\year}% Store date
\paragraph*{\kern-2ex\quad #1, \syldate{\today} - \AdvanceDate[4]\syldate{\today}:}% Set heading  \quad #1
%  \setbox1=\hbox{\shortdayofweekname{\getdateday{mydate}}{\getdatemonth{mydate}}{\getdateyear{mydate}}}%
\ifdim\wd1=\wd\MONDAY
\AdvanceDate[7]
\else
\AdvanceDate[7]
\fi%
}
\usepackage{setspace}
\usepackage{multicol}
%\usepackage{indentfirst}
\usepackage{fancyhdr,lastpage}
\usepackage{url}
\pagestyle{fancy}
\usepackage{hyperref}
\usepackage{lastpage}
\usepackage{amsmath}
\usepackage{layout}
\usepackage{booktabs}
\usepackage{array}
\newcolumntype{L}[1]{>{\raggedright\let\newline\\\arraybackslash\hspace{0pt}}m{#1}}
\newcolumntype{C}[1]{>{\centering\let\newline\\\arraybackslash\hspace{0pt}}m{#1}}
\newcolumntype{R}[1]{>{\raggedleft\let\newline\\\arraybackslash\hspace{0pt}}m{#1}}

\lhead{}
\chead{}
%%%%%%%%%%%%%%%%%%%%%%%%%%%%%%%%%%%%%%%%%%%%%%%%%%%%%%%%%%%%%%

% Modify header here %%%%%%%%%%%%%%%%%%%%%%%%%%%%%%%%%%%%%%%%%
\rhead{\footnotesize UNIVERSIDAD DE ANTIOQUIA - Facultad de Ingeniería}

%%%%%%%%%%%%%%%%%%%%%%%%%%%%%%%%%%%%%%%%%%%%%%%%%%%%%%%%%%%%%%
% Don't touch this %%%%%%%%%%%%%%%%%%%%%%%%%%%%%%%%%%%%%%%%%%%
\lfoot{}
\cfoot{\small \thepage/\pageref*{LastPage}}
\rfoot{}

\usepackage{array, xcolor}
\usepackage{color,hyperref}
\definecolor{clemsonorange}{HTML}{EA6A20}
\hypersetup{colorlinks,breaklinks,linkcolor=clemsonorange,urlcolor=clemsonorange,anchorcolor=clemsonorange,citecolor=black}

\begin{document}

\maketitle
%\blankline

\begin{tabular*}{.93\textwidth}{@{\extracolsep{\fill}}lr}
\hline\\
%%%%%%%%%%%%%%%%%%%%%%%%%%%%%%%%%%%%%%%%%%%%%%%%%%%%%%%%%%%%%%

% Modify information %%%%%%%%%%%%%%%%%%%%%%%%%%%%%%%%%%%%%%%%%
%E-mail: \texttt{username@ncsu.edu} & Web: \href{www4.ncsu.edu/~username}{\tt\bf www4.ncsu.edu/~username}  \\
\textbf{Semestre:} 5 & \textbf{Semanas:} 16
\\
\textbf{Área:} Básica de Ingeniería &    \textbf{Créditos:} 5 
\\ & \\
\textbf{Programas a los cuales se ofrece:} Ingeniería de Telecomunicaciones
\\ & \\
\hline
\end{tabular*}

\vspace{5 mm}

%%%%%%%%%%%%%%%%%%%%%%%%%%%%%%%%%%%%%%%%%%%%
\section*{Propósito del curso}

Vacio

%\bigskip

%\noindent New paragraph. Bla bla bla ...

%%%%%%%%%%%%%%%%%%%%%%%%%%%%%%%%%%%%%%%%%%%%
\section*{Justificación}

Vacio

%%%%%%%%%%%%%%%%%%%%%%%%%%%%%%%%%%%%%%%%%%%%
\section*{Prerrequisitos/Correquisitos}
\begin{description}
\item [Prerrequisitos:] Ninguno
\item[Correquisitos:] Fundamentos de Electrónica Análoga
\end{description}

%%%%%%%%%%%%%%%%%%%%%%%%%%%%%%%%%%%%%%%%%%%%
\section*{Objetivos}

\subsection*{General}

\begin{itemize}
\item No ha agregado este campo 
\end{itemize}

\subsection*{Específicos}

\begin{itemize}
\item Entender como funciona la aritmética en los computadores: complemento a 2, punto flotante, etc. \item Entender los conceptos de Conjunto de Instrucciones (ISA), y Microarquitectura. \item Entender la implementación de cada una de las estructuras de un procesador funcional. \item Entender la interacción entre el HW y el SW en sistemas computacionales. \item Aprender a programar en assembler. \item Entender la técnica del pipeline, en particular, como funciona en los procesadores modernos. \item Realizar una aplicación de mediana complejidad en FPGA. 
\end{itemize}

%%%%%%%%%%%%%%%%%%%%%%%%%%%%%%%%%%%%%%%%%%%%
\section*{Contenido Resumido}

\begin{itemize}
\item No ha agregado este campo 
\end{itemize}

%%%%%%%%%%%%%%%%%%%%%%%%%%%%%%%%%%%%%%%%%%%%
\section*{Unidades}
\noindent 
No ha agregado este campo

%%%%%%%%%%%%%%%%%%%%%%%%%%%%%%%%%%%%%%%%%%%%
\section*{Metodología}

Vacio

%%%%%%%%%%%%%%%%%%%%%%%%%%%%%%%%%%%%%%%%%%%%
\section*{Evaluación}
\noindent \begin{tabular}{R{0.5\textwidth} C{0.2\textwidth} C{0.2\textwidth}}
	\toprule
	\textbf{Actividad} & \textbf{Porcentaje} & \textbf{Fecha} \\
	\midrule
	vacio & vacio & vacio \\
	\midrule
\end{tabular}
\\
%%%%%%%%%%%%%%%%%%%%%%%%%%%%%%%%%%%%%%%%%%%%
\section*{Actividades de asistencia obligatoria}

Vacio

%%%%%%%%%%%%%%%%%%%%%%%%%%%%%%%%%%%%%%%%%%%%
\section*{Bibliografía}

\subsection*{Básica}

\begin{itemize}
\item No ha agregado este campo 
\end{itemize}

\subsection*{Complementaria}

\begin{itemize}
\item No ha agregado este campo 
\end{itemize}

\end{document}
