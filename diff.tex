% Don't touch this %%%%%%%%%%%%%%%%%%%%%%%%%%%%%%%%%%%%%%%%%%%
%DIF LATEXDIFF DIFFERENCE FILE
%DIF DEL /home/nicolas/CursoDjango/Microcurriculos/Microcurriculos_Udea/salida.tex    Thu Apr 23 17:58:47 2020
%DIF ADD /home/nicolas/CursoDjango/Microcurriculos/Microcurriculos_Udea/salida2.tex   Thu Apr 23 17:58:47 2020
\documentclass[11pt]{article}
\usepackage[utf8]{inputenc}
\usepackage{fullpage}
\usepackage[left=1in,top=1in,right=1in,bottom=1in,headheight=3ex,headsep=3ex]{geometry}
\usepackage{graphicx}
\usepackage{float}
\usepackage[spanish]{babel}

\newcommand{\blankline}{\quad\pagebreak[2]}
%%%%%%%%%%%%%%%%%%%%%%%%%%%%%%%%%%%%%%%%%%%%%%%%%%%%%%%%%%%%%%

% Modify Course title, instructor name, semester here %%%%%%%%

\title{2536100: Descubriendo la Física}
\author{\begin{tabular} {c|c|c|c|c|c} Teórico & \textbf{0} & Teórico-Práctico & \textbf{3} & Práctico & \textbf{0}\end{tabular}
\\
\begin{tabular} {c|c|c|c|c|c} Validable & \textbf{Si} & Habilitable & \textbf{Si} & Clasificable & \textbf{Si}\end{tabular}}
\date{2016-1 a 2016-2 , 2017-2 , 2018-2 a 2019-1}

%%%%%%%%%%%%%%%%%%%%%%%%%%%%%%%%%%%%%%%%%%%%%%%%%%%%%%%%%%%%%%

% Don't touch this %%%%%%%%%%%%%%%%%%%%%%%%%%%%%%%%%%%%%%%%%%%
\usepackage[sc]{mathpazo}
\linespread{1.05} % Palatino needs more leading (space between lines)
\usepackage[T1]{fontenc}
\usepackage[mmddyyyy]{datetime}% http://ctan.org/pkg/datetime
\usepackage{advdate}% http://ctan.org/pkg/advdate
\newdateformat{syldate}{\twodigit{\THEMONTH}/\twodigit{\THEDAY}}
\newsavebox{\MONDAY}\savebox{\MONDAY}{Mon}% Mon
\newcommand{\week}[1]{%
%  \cleardate{mydate}% Clear date
% \newdate{mydate}{\the\day}{\the\month}{\the\year}% Store date
\paragraph*{\kern-2ex\quad #1, \syldate{\today} - \AdvanceDate[4]\syldate{\today}:}% Set heading  \quad #1
%  \setbox1=\hbox{\shortdayofweekname{\getdateday{mydate}}{\getdatemonth{mydate}}{\getdateyear{mydate}}}%
\ifdim\wd1=\wd\MONDAY
\AdvanceDate[7]
\else
\AdvanceDate[7]
\fi%
}
\usepackage{setspace}
\usepackage{multicol}
%\usepackage{indentfirst}
\usepackage{fancyhdr,lastpage}
\usepackage{url}
\pagestyle{fancy}
\usepackage{hyperref}
\usepackage{lastpage}
\usepackage{amsmath}
\usepackage{layout}
\usepackage{booktabs}
\usepackage{array}
\newcolumntype{L}[1]{>{\raggedright\let\newline\\\arraybackslash\hspace{0pt}}m{#1}}
\newcolumntype{C}[1]{>{\centering\let\newline\\\arraybackslash\hspace{0pt}}m{#1}}
\newcolumntype{R}[1]{>{\raggedleft\let\newline\\\arraybackslash\hspace{0pt}}m{#1}}

\lhead{}
\chead{}
%%%%%%%%%%%%%%%%%%%%%%%%%%%%%%%%%%%%%%%%%%%%%%%%%%%%%%%%%%%%%%

% Modify header here %%%%%%%%%%%%%%%%%%%%%%%%%%%%%%%%%%%%%%%%%
\rhead{\footnotesize UNIVERSIDAD DE ANTIOQUIA - Facultad de Ingeniería}

%%%%%%%%%%%%%%%%%%%%%%%%%%%%%%%%%%%%%%%%%%%%%%%%%%%%%%%%%%%%%%
% Don't touch this %%%%%%%%%%%%%%%%%%%%%%%%%%%%%%%%%%%%%%%%%%%
\lfoot{}
\cfoot{\small \thepage/\pageref*{LastPage}}
\rfoot{}

\usepackage{array, xcolor}
\usepackage{color,hyperref}
\definecolor{clemsonorange}{HTML}{EA6A20}
\hypersetup{colorlinks,breaklinks,linkcolor=clemsonorange,urlcolor=clemsonorange,anchorcolor=clemsonorange,citecolor=black}
%DIF PREAMBLE EXTENSION ADDED BY LATEXDIFF
%DIF UNDERLINE PREAMBLE %DIF PREAMBLE
\RequirePackage[normalem]{ulem} %DIF PREAMBLE
\RequirePackage{color}\definecolor{RED}{rgb}{1,0,0}\definecolor{BLUE}{rgb}{0,0,1} %DIF PREAMBLE
\providecommand{\DIFaddtex}[1]{{\protect\color{blue}\uwave{#1}}} %DIF PREAMBLE
\providecommand{\DIFdeltex}[1]{{\protect\color{red}\sout{#1}}}                      %DIF PREAMBLE
%DIF SAFE PREAMBLE %DIF PREAMBLE
\providecommand{\DIFaddbegin}{} %DIF PREAMBLE
\providecommand{\DIFaddend}{} %DIF PREAMBLE
\providecommand{\DIFdelbegin}{} %DIF PREAMBLE
\providecommand{\DIFdelend}{} %DIF PREAMBLE
%DIF FLOATSAFE PREAMBLE %DIF PREAMBLE
\providecommand{\DIFaddFL}[1]{\DIFadd{#1}} %DIF PREAMBLE
\providecommand{\DIFdelFL}[1]{\DIFdel{#1}} %DIF PREAMBLE
\providecommand{\DIFaddbeginFL}{} %DIF PREAMBLE
\providecommand{\DIFaddendFL}{} %DIF PREAMBLE
\providecommand{\DIFdelbeginFL}{} %DIF PREAMBLE
\providecommand{\DIFdelendFL}{} %DIF PREAMBLE
%DIF HYPERREF PREAMBLE %DIF PREAMBLE
\providecommand{\DIFadd}[1]{\texorpdfstring{\DIFaddtex{#1}}{#1}} %DIF PREAMBLE
\providecommand{\DIFdel}[1]{\texorpdfstring{\DIFdeltex{#1}}{}} %DIF PREAMBLE
\newcommand{\DIFscaledelfig}{0.5}
%DIF HIGHLIGHTGRAPHICS PREAMBLE %DIF PREAMBLE
\RequirePackage{settobox} %DIF PREAMBLE
\RequirePackage{letltxmacro} %DIF PREAMBLE
\newsavebox{\DIFdelgraphicsbox} %DIF PREAMBLE
\newlength{\DIFdelgraphicswidth} %DIF PREAMBLE
\newlength{\DIFdelgraphicsheight} %DIF PREAMBLE
% store original definition of \includegraphics %DIF PREAMBLE
\LetLtxMacro{\DIFOincludegraphics}{\includegraphics} %DIF PREAMBLE
\newcommand{\DIFaddincludegraphics}[2][]{{\color{blue}\fbox{\DIFOincludegraphics[#1]{#2}}}} %DIF PREAMBLE
\newcommand{\DIFdelincludegraphics}[2][]{% %DIF PREAMBLE
\sbox{\DIFdelgraphicsbox}{\DIFOincludegraphics[#1]{#2}}% %DIF PREAMBLE
\settoboxwidth{\DIFdelgraphicswidth}{\DIFdelgraphicsbox} %DIF PREAMBLE
\settoboxtotalheight{\DIFdelgraphicsheight}{\DIFdelgraphicsbox} %DIF PREAMBLE
\scalebox{\DIFscaledelfig}{% %DIF PREAMBLE
\parbox[b]{\DIFdelgraphicswidth}{\usebox{\DIFdelgraphicsbox}\\[-\baselineskip] \rule{\DIFdelgraphicswidth}{0em}}\llap{\resizebox{\DIFdelgraphicswidth}{\DIFdelgraphicsheight}{% %DIF PREAMBLE
\setlength{\unitlength}{\DIFdelgraphicswidth}% %DIF PREAMBLE
\begin{picture}(1,1)% %DIF PREAMBLE
\thicklines\linethickness{2pt} %DIF PREAMBLE
{\color[rgb]{1,0,0}\put(0,0){\framebox(1,1){}}}% %DIF PREAMBLE
{\color[rgb]{1,0,0}\put(0,0){\line( 1,1){1}}}% %DIF PREAMBLE
{\color[rgb]{1,0,0}\put(0,1){\line(1,-1){1}}}% %DIF PREAMBLE
\end{picture}% %DIF PREAMBLE
}\hspace*{3pt}}} %DIF PREAMBLE
} %DIF PREAMBLE
\LetLtxMacro{\DIFOaddbegin}{\DIFaddbegin} %DIF PREAMBLE
\LetLtxMacro{\DIFOaddend}{\DIFaddend} %DIF PREAMBLE
\LetLtxMacro{\DIFOdelbegin}{\DIFdelbegin} %DIF PREAMBLE
\LetLtxMacro{\DIFOdelend}{\DIFdelend} %DIF PREAMBLE
\DeclareRobustCommand{\DIFaddbegin}{\DIFOaddbegin \let\includegraphics\DIFaddincludegraphics} %DIF PREAMBLE
\DeclareRobustCommand{\DIFaddend}{\DIFOaddend \let\includegraphics\DIFOincludegraphics} %DIF PREAMBLE
\DeclareRobustCommand{\DIFdelbegin}{\DIFOdelbegin \let\includegraphics\DIFdelincludegraphics} %DIF PREAMBLE
\DeclareRobustCommand{\DIFdelend}{\DIFOaddend \let\includegraphics\DIFOincludegraphics} %DIF PREAMBLE
\LetLtxMacro{\DIFOaddbeginFL}{\DIFaddbeginFL} %DIF PREAMBLE
\LetLtxMacro{\DIFOaddendFL}{\DIFaddendFL} %DIF PREAMBLE
\LetLtxMacro{\DIFOdelbeginFL}{\DIFdelbeginFL} %DIF PREAMBLE
\LetLtxMacro{\DIFOdelendFL}{\DIFdelendFL} %DIF PREAMBLE
\DeclareRobustCommand{\DIFaddbeginFL}{\DIFOaddbeginFL \let\includegraphics\DIFaddincludegraphics} %DIF PREAMBLE
\DeclareRobustCommand{\DIFaddendFL}{\DIFOaddendFL \let\includegraphics\DIFOincludegraphics} %DIF PREAMBLE
\DeclareRobustCommand{\DIFdelbeginFL}{\DIFOdelbeginFL \let\includegraphics\DIFdelincludegraphics} %DIF PREAMBLE
\DeclareRobustCommand{\DIFdelendFL}{\DIFOaddendFL \let\includegraphics\DIFOincludegraphics} %DIF PREAMBLE
%DIF END PREAMBLE EXTENSION ADDED BY LATEXDIFF

\begin{document}

\maketitle
%\blankline

\begin{tabular*}{.93\textwidth}{@{\extracolsep{\fill}}lr}
\hline\\
%%%%%%%%%%%%%%%%%%%%%%%%%%%%%%%%%%%%%%%%%%%%%%%%%%%%%%%%%%%%%%

% Modify information %%%%%%%%%%%%%%%%%%%%%%%%%%%%%%%%%%%%%%%%%
%E-mail: \texttt{username@ncsu.edu} & Web: \href{www4.ncsu.edu/~username}{\tt\bf www4.ncsu.edu/~username}  \\
\textbf{Semestre:} 1 & \textbf{Semanas:} 16
\\
\textbf{Área:} Ciencias Básicas &    \textbf{Créditos:} 4 
\\ & \\
\textbf{Programas a los cuales se ofrece:} Ingeniería de Telecomunicaciones
\\ & \\
\hline
\end{tabular*}

\vspace{5 mm}

%%%%%%%%%%%%%%%%%%%%%%%%%%%%%%%%%%%%%%%%%%%%
\section*{Propósito del curso}

Proporcionar al estudiante los conocimientos y las técnicas operativas básicas requeridas para la resolución de problemas matemáticos que surgen en el álgebra y la trigonometría.

%\bigskip

%\noindent New paragraph. Bla bla bla ...

%%%%%%%%%%%%%%%%%%%%%%%%%%%%%%%%%%%%%%%%%%%%
\section*{Justificación}

Proporcionar al estudiante los conocimientos y las técnicas operativas básicas requeridas para la resolución de problemas matemáticos que surgen en el álgebra y la trigonometría. El curso es fundamental, pues proporciona bases sólidas indispensables para abordar los cursos posteriores de cálculo.

%%%%%%%%%%%%%%%%%%%%%%%%%%%%%%%%%%%%%%%%%%%%
\section*{Prerrequisitos/Correquisitos}
\begin{description}
\item [Prerrequisitos:] Ninguno
\item[Correquisitos:] Ninguno
\end{description}

%%%%%%%%%%%%%%%%%%%%%%%%%%%%%%%%%%%%%%%%%%%%
\section*{Objetivos}

\subsection*{General}

\begin{itemize}
\item \DIFdelbegin \DIFdel{- Contribuir al desarrollo del intelecto y de la capacidad analítica del estudiante, potenciando facultades cognitivas de orden superior y la abstracción. - Facilitar la comprensión de las leyes de la naturaleza y los conceptos fundamentales en los que se basan los métodos para el análisis y diseño de sistemas en ingeniería. - Formar en el estudiante las reglas de la demostración o refutación rigurosa y de la explicación válida. - Establecer un lenguaje común, básico, para comunicarse con otros profesionales y para adelantar estudios e investigaciones avanzadas. 
}\DIFdelend \DIFaddbegin \DIFadd{general 1 }\item \DIFadd{general 2 
}\DIFaddend \end{itemize}

\subsection*{Específicos}

\begin{itemize}
\item \DIFdelbegin \DIFdel{Una vez aprobada la asignatura, el estudiante debe estar en capacidad de utilizar los conceptos básicos del Álgebra y la trigonometría en la solución de problemas matemáticos de interés en el área de ingeniería y en particular: - Poder realizar en cualquier conjunto numérico: factorización, potenciación y solución de inecuaciones. Descomponer una fracción racional en fracciones parciales. - Operar adecuadamente con funciones y ecuaciones polinómicas y las correspondientes aplicaciones. Analizar otros tipos de funciones algebraicas. - Operar con funciones exponenciales y logarítmicas y sus respectivas aplicaciones. - Resolver problemas típicos de ingeniería empleando elementos fundamentales de trigonometría. - Operar con los números complejos y sus diferentes representaciones. 
}\DIFdelend \DIFaddbegin \DIFadd{especifico 1 
}\DIFaddend \end{itemize}

%%%%%%%%%%%%%%%%%%%%%%%%%%%%%%%%%%%%%%%%%%%%
\section*{Contenido Resumido}

\begin{itemize}
\item \DIFdelbegin \DIFdel{- Unidad No.1: Conceptos fundamentales. Ecuaciones,  desigualdades y Funciones. - Unidad No.2: Polinomios y funciones racionales. - Unidad No.3: Funciones Inversas, exponenciales y logarítmicas. - Unidad No.4: Funciones trigonométricas. Trigonometría analítica. - Unidad No.5: Trigonometría 
}\DIFdelend \DIFaddbegin \DIFadd{item 1 
}\DIFaddend \end{itemize}

%%%%%%%%%%%%%%%%%%%%%%%%%%%%%%%%%%%%%%%%%%%%
\section*{Unidades}
\noindent 
\begin{tabular}{R{0.16\textwidth} L{0.7\textwidth}} 
 \\ 
\toprule \textbf{Unidad No. 1} & Conceptos fundamentales. Ecuaciones, desigualdades y  Funciones. 
 \\ 
\midrule\textbf{Subtemas} & 
\begin{description}
 \item \DIFdelbegin \DIFdel{CLASE }\DIFdelend \DIFaddbegin \DIFadd{subtema 1 unidad }\DIFaddend 1 
\DIFdelbegin \DIFdel{: Presentación del mapa conceptual del curso. CLASE 2: Razones y proporciones. Conjuntos numéricos. Ejercicios de aplicación. CLASE 3: Progresiones aritméticas y geométricas. Sumatoria y productoria. Ejercicios de aplicación. CLASE 4: Potenciación y radicación. Leyes de los exponentes y los radicales. Racionalización. CLASE 5: Ejercicios de aplicación sobre potenciación y radicación. CLASE 6 y 7: Polinomios. Operaciones básicas. Productos notables y factorización. Binomio de Newton con exponente natural y triángulo de pascal. Ejercicios de aplicación. CLASE 8: 1ra EVALUACIÓN DEL 20%DIF <  
}\DIFdelend \end{description}
 \\ 
\textbf{Semanas} & 4 
\end{tabular} 
 \\ 
 \begin{tabular}{R{0.16\textwidth} L{0.7\textwidth}} 
 \\ 
\toprule \textbf{Unidad No. 2} & Polinomios y funciones racionales. 
 \\ 
\midrule\textbf{Subtemas} & 
\begin{description}
 \item \DIFdelbegin \DIFdel{CLASE 9: Funciones y ecuaciones polinómicas. El polinomio cuadrático. Raíces de una ecuación cuadrática. Ecuaciones reducibles a cuadráticas. Ejercicios de aplicación. CLASE 10: Ejercicios y problemas de aplicación de funciones cuadráticas y de sistemas de ecuaciones dos por dos. CLASE 11: Ejercicios de aplicación de funciones cuadráticas: Problemas de velocidad y tiempo. Llenado de tanques. CLASE 12: Polinomios de grado superior. Teoremas del residuo y del factor. Ejercicios de aplicación y gráficas. CLASE 13: La división sintética. Teorema de los ceros racionales. Ley de los signos de descartes. Ejercicios de aplicación. CLASE 14: 2da EVALUACIÓN DEL 20%DIF <  
}\DIFdelend \DIFaddbegin \DIFadd{subtema 1 unidad 2 
}\DIFaddend \end{description}
 \\ 
\textbf{Semanas} & 3 
\end{tabular} 
 \\ 
 \begin{tabular}{R{0.16\textwidth} L{0.7\textwidth}} 
 \\ 
\toprule \textbf{Unidad No. 3} & \DIFdelbegin \DIFdel{Funciones Inversas, exponenciales y logarítmicas. 
 }%DIFDELCMD < \\ 
%DIFDELCMD < \midrule%%%
\textbf{\DIFdel{Subtemas}} %DIFAUXCMD
%DIFDELCMD < & 
%DIFDELCMD < \begin{description}
\begin{description}%DIFAUXCMD
%DIFDELCMD <  \item %%%
\item%DIFAUXCMD
\DIFdel{CLASE 15: Operaciones con fracciones. Simplificación de fracciones. Fracciones racionales. Fracción continuada. Ejercicios de aplicación. CLASE 16 y 17: Descomposición en fracciones parciales. Ejercicios de aplicación. CLASE 18: Crecimiento y decrecimiento. La función exponencial. Ejercicios CLASE 19: Función logarítmica. Logaritmos. Ecuaciones exponenciales y logarítmicas. Ejercicios de aplicación. CLASE 20: 3ra EVALUACIÓN DEL 20%DIF <  
}
\end{description}%DIFAUXCMD
%DIFDELCMD < \end{description}
%DIFDELCMD <  \\ 
%DIFDELCMD < %%%
\textbf{\DIFdel{Semanas}} %DIFAUXCMD
%DIFDELCMD < & %%%
\DIFdel{3 
}%DIFDELCMD < \end{tabular} 
%DIFDELCMD <  \\ 
%DIFDELCMD <  \begin{tabular}{R{0.16\textwidth} L{0.7\textwidth}} 
%DIFDELCMD <  \\ 
%DIFDELCMD < \toprule %%%
\textbf{\DIFdel{Unidad No. 4}} %DIFAUXCMD
%DIFDELCMD < & %%%
\DIFdel{Funciones trigonométricas. Trigonometría analítica. 
 }%DIFDELCMD < \\ 
%DIFDELCMD < \midrule%%%
\textbf{\DIFdel{Subtemas}} %DIFAUXCMD
%DIFDELCMD < & 
%DIFDELCMD < \begin{description}
\begin{description}%DIFAUXCMD
%DIFDELCMD <  \item %%%
\item%DIFAUXCMD
\DIFdel{CLASE 21: Trigonometría del triángulo rectángulo. Introducción. Medición de ángulos. Ángulos notables. Ejercicios de aplicación. CLASE 22: Resolución de triángulos (incluyendo: dados tres lados hallar los tres ángulos). Teorema del seno. Teorema del coseno. Resolución de triángulos oblicuángulos. Ejercicios y problemas de solución de triángulos. CLASE 23: Ejercicios y problemas de solución de triángulos. CLASE 24: Trigonometría del círculo. Funciones circulares. Identidades fundamentales. Gráficas de las funciones trigonométricas. Funciones periódicas. Ejercicios aplicaciones. CLASE 25: Funciones de la suma y diferencia de ángulos. Ángulo doble y ángulo medio. Transformación de sumas y diferencias en productos. Transformación de productos en sumas. Ejercicios de aplicación. CLASE 26: 4 ta. EVALUACIÓN DEL 20%DIF <  
}
\end{description}%DIFAUXCMD
%DIFDELCMD < \end{description}
%DIFDELCMD <  \\ 
%DIFDELCMD < %%%
\textbf{\DIFdel{Semanas}} %DIFAUXCMD
%DIFDELCMD < & %%%
\DIFdel{3 
}%DIFDELCMD < \end{tabular} 
%DIFDELCMD <  \\ 
%DIFDELCMD <  \begin{tabular}{R{0.16\textwidth} L{0.7\textwidth}} 
%DIFDELCMD <  \\ 
%DIFDELCMD < \toprule %%%
\textbf{\DIFdel{Unidad No. 5}} %DIFAUXCMD
%DIFDELCMD < & %%%
\DIFdel{Trigonometría 
 }\DIFdelend \DIFaddbegin \DIFadd{illuminati 
 }\DIFaddend \\ 
\midrule\textbf{Subtemas} & 
\begin{description}
 \item \DIFdelbegin \DIFdel{CLASE 27: Funciones trigonométricas inversas. Identidades y Ecuaciones Trigonométricas. Ejercicios de aplicación. CLASE 28: Demostración de identidades y Ecuaciones Trigonométricas. Ejercicios de aplicación. CLASE 29: Los números complejos. Introducción. Propiedades. Forma estándar de los números complejos. El plano de Argand. Ejercicios de aplicación. CLASE 30: Forma polar y exponencial de los números complejos. Operaciones fundamentales. Ejercicios de aplicación. CLASE 31: Potencias y raíces de números complejos. Polos y ceros. Ejercicios de aplicación. CLASE 32: EXAMEN FINAL DEL 20%DIF <  
}\DIFdelend \DIFaddbegin \DIFadd{a 
}\DIFaddend \end{description}
 \\ 
\textbf{Semanas} & \DIFdelbegin \DIFdel{3 
}\DIFdelend \DIFaddbegin \DIFadd{1 
}\DIFaddend \end{tabular} 
 \\ 


%%%%%%%%%%%%%%%%%%%%%%%%%%%%%%%%%%%%%%%%%%%%
\section*{Metodología}

- Exposición magistral del docente. - Talleres semanales dirigidos por el monitor.

%%%%%%%%%%%%%%%%%%%%%%%%%%%%%%%%%%%%%%%%%%%%
\section*{Evaluación}
\noindent \begin{tabular}{R{0.5\textwidth} C{0.2\textwidth} C{0.2\textwidth}}
	\toprule
	\textbf{Actividad} & \textbf{Porcentaje} & \textbf{Fecha} \\
	\\
	\midrule
	Sesiones 1, 2, 3, 4, 5. 6, 7 & 20 & 2019-01-12 \\ Sesiones 9, 10, 11, 12, 13 & 20 & 2019-02-12 \\ 
	\DIFdelbegin \DIFdel{Sesiones 15, 16,17, 18, 19 }%DIFDELCMD < & %%%
\DIFdel{20 }%DIFDELCMD < & %%%
\DIFdel{2019-03-12 }%DIFDELCMD < \\ %%%
\DIFdel{Sesiones 21, 22, 23, 24,25 }%DIFDELCMD < & %%%
\DIFdel{20 }%DIFDELCMD < & %%%
\DIFdel{2019-04-12 }%DIFDELCMD < \\ %%%
\DIFdel{Sesiones 27, 28, 29, 30, 31 }%DIFDELCMD < & %%%
\DIFdel{20 }%DIFDELCMD < & %%%
\DIFdel{2019-05-12 }%DIFDELCMD < \\ 
%DIFDELCMD < 	%%%
\DIFdelend \\
	\midrule
\end{tabular}
\\
%%%%%%%%%%%%%%%%%%%%%%%%%%%%%%%%%%%%%%%%%%%%
\section*{Actividades de asistencia obligatoria}

Exámenes parciales

%%%%%%%%%%%%%%%%%%%%%%%%%%%%%%%%%%%%%%%%%%%%
\section*{Bibliografía}

\subsection*{Básica}

\begin{itemize}
\item \DIFdelbegin \DIFdel{Texto guía: Álgebra y Trigonometría. Editado por Ude@. Tercera edición Benjamín Buriticá Trujillo. http://docencia.udea.edu.co/cen/AlgebraTrigonometria 
}\DIFdelend \DIFaddbegin \DIFadd{basica 1 
}\DIFaddend \end{itemize}

\subsection*{Complementaria}

\begin{itemize}
\item \DIFdelbegin \DIFdel{Bibliografía complementaria: Textos de consulta: - Zill,D. y Dewar, J. Algebra y trigonometría. McGraw-Hill. - Diez Luis H. Matemáticas Operativas. - Leithold Louis. Algebra y Trigonometría - Walter Fleming y Dale Varbeg. Algebra y Trigonometría con Geometría - Analítica. Prentice-Hall. - James Stewart. Rotar Redhin. Saleem Watson Precálculo. Thomson Learning. Tercera Edición 
}\DIFdelend \DIFaddbegin \DIFadd{complemen 1 
}\DIFaddend \end{itemize}

\end{document}
